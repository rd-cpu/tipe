\documentclass[11pt,a4paper]{article}

\usepackage[utf8]{inputenc}
\usepackage[T1]{fontenc}
\usepackage[french]{babel}
\usepackage{amsmath, amssymb}
\usepackage{geometry}
\usepackage{setspace}
\usepackage{hyperref}

\geometry{margin=2.5cm}
\onehalfspacing

\title{\textbf{Analyse, simulation numérique et expérimentation\\
des oscillateurs de Van der Pol}}
\author{}
\date{}

\begin{document}
\maketitle

\section*{Liste des membres du groupe}
\begin{itemize}
    \item HARRACHE Gwevenn
    \item COMMUNAL Hugo
\end{itemize}

\section*{Positionnements thématiques}
\begin{itemize}
    \item Physique (électronique)
    \item Mathématiques (analyse, équations différentielles non linéaires)
    \item Informatique (informatique pratique, Python)
\end{itemize}

\section*{Mots-clés}
\begin{itemize}
    \item Simulation numérique
    \item Oscillateur non linéaire
    \item Cycle limite
    \item Amortissement négatif
    \item Équation différentielle
    \item Attracteur
\end{itemize}

\section*{Bibliographie commentée}

Introduit par le physicien et ingénieur néerlandais Balthasar Van der Pol dans les années 1920, 
l’oscillateur de Van der Pol est un système dynamique non linéaire emblématique permettant de 
mettre en œuvre le théorème de Poincaré--Bendixson \cite{poincare}. Initialement conçu pour 
modéliser des phénomènes électriques dans les circuits à lampe triode, cet oscillateur a par la 
suite trouvé des applications dans des domaines tels que la biologie (modélisation du rythme 
cardiaque), la mécanique, la sismologie ou encore les systèmes chaotiques.

L’étude de ce système permet d’aborder les notions de non-linéarité, d’auto-oscillations et de 
cycles limites.

\subsection*{Équation de Van der Pol}

L’équation différentielle dite de Van der Pol s’écrit :
\[
\frac{d^{2}x(t)}{dt^{2}}
- \varepsilon \omega_{0} \left(1 - x^{2}(t)\right)\frac{dx(t)}{dt}
+ \omega_{0}^{2}x(t) = 0
\]

avec :
\begin{itemize}
    \item $\varepsilon$ : paramètre de non-linéarité modulant l’intensité de l’amortissement négatif,
    \item $\omega_{0}$ : pulsation propre du système.
\end{itemize}

Cette équation n’étant pas résolvable analytiquement du fait du coefficient non constant du terme 
d’ordre un, l’utilisation de simulations numériques est nécessaire afin d’obtenir des solutions 
approchées de $x(t)$.

L’existence des comportements limites de cet oscillateur peut être prouvée par le théorème de 
Poincaré--Bendixson, qui stipule que soit $x(t)$ converge vers une limite, soit son comportement 
asymptotique est une fonction périodique appelée \textit{cycle limite}. Ce phénomène constitue un 
exemple typique d’oscillations auto-entretenues.

\subsection*{Approche numérique et expérimentale}

Il est possible de mettre en œuvre des systèmes électriques utilisant des amplificateurs 
opérationnels, des condensateurs et des bobines, mais également de simuler ce comportement à 
l’aide d’outils numériques. Une résolution utilisant des langages de programmation tels que 
Python permet une approche simple et efficace du phénomène.

Les simulations permettent notamment d’obtenir des résultats concernant :
\begin{itemize}
    \item les cycles limites,
    \item la durée du régime transitoire,
    \item la période des oscillations.
\end{itemize}

Une approche visuelle est également possible grâce aux diagrammes de phase représentant 
$\frac{dx}{dt}$ en fonction de $x$. Le caractère attractif du cycle limite permet de prédire 
certains comportements à partir des isoclines.

\subsection*{Oscillateur de Van der Pol forcé}

Une seconde version de l’équation, dite \textit{forcée}, s’écrit :
\[
\frac{d^{2}x(t)}{dt^{2}}
- \varepsilon \omega_{0} \left(1 - x^{2}(t)\right)\frac{dx(t)}{dt}
+ \omega_{0}^{2}x(t)
= \omega_{0}^{2}X\cos(\omega t)
\]

Cette version conduit à des comportements chaotiques, sensibles aux conditions initiales et 
non prévisibles à long terme, étudiés dans le cadre de la théorie du chaos déterministe.

\section*{Problématique retenue}

Comment les oscillateurs de Van der Pol permettent-ils d’illustrer l’existence et les 
caractéristiques de cycles limites dans des systèmes dynamiques non linéaires ?

\section*{Objectifs du TIPE}
\begin{itemize}
    \item Simulations numériques pour différentes valeurs de paramètres
    \item Mise en place d’un dispositif expérimental sous forme de circuit électrique
    \item Interprétation et exploitation des résultats expérimentaux
    \item Preuve de l’existence du cycle limite à l’aide du théorème de Poincaré--Bendixson
\end{itemize}

\section*{Références}
\begin{thebibliography}{9}

\bibitem{poincare}
Théorème de Poincaré--Bendixson — Wikipédia

\bibitem{gleick}
J. Gleick, \textit{Chaos: Making a New Science}, Flammarion, 1988.

\bibitem{vdp}
B. Van der Pol, J. Van der Mark, \textit{The Heartbeat considered as a Relaxation Oscillation}, 
Philosophical Magazine, 1928.

\bibitem{moon}
F. C. Moon, \textit{Chaotic Vibrations}, Wiley, 1992.

\end{thebibliography}

\end{document}
