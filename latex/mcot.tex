\documentclass[11pt,a4paper]{article}

\usepackage[utf8]{inputenc}
\usepackage[T1]{fontenc}
\usepackage[french]{babel}
\usepackage{amsmath, amssymb}
\usepackage{geometry}
\usepackage{setspace}
\usepackage{hyperref}

\geometry{margin=2.5cm}
\onehalfspacing

\title{\textbf{Mise en place d'un cryptosystème d'EL Gamal sur les courbes elliptiques et étude comparative de sa sécurité.}}
\date{}

\begin{document}
\maketitle

\section*{Liste des membres du groupe}
\begin{itemize}
    \item DESNEUX Raphaël(snif il s'en va....) 
    \item COMMUNAL Hugo
\end{itemize}

\section*{Positionnements thématiques}
\begin{itemize}
    \item Mathématiques (Algèbre, Théorie des groupes, Coubres elliptiques)
    \item Informatique (informatique théorique, Python)
\end{itemize}
\vspace{1em}

\noindent
\begin{minipage}[t]{0.45\textwidth}
%\textbf{Mots-clés (en français)}\\[0.3em]
\section*{Mots-clés (en français)}
\text{- Cryptographie}\\
\text{- Cryptanalyse}\\
\text{- Groupes cycliques}\\
\text{- Courbes elliptiques}\\
\text{- Logarithme discret}\\
\text{- Complexité algorithmique}\\
\text{- Calcul numérique}
\end{minipage}
\hfill
\begin{minipage}[t]{0.45\textwidth}
%\textbf{Mots-clés (en anglais)}\\[0.3em]
\section*{Mots-clés (en anglais)}
\text{- Cryptography}\\
\text{- Cryptanalysis}\\
\text{- Cyclic groups}\\
\text{- Elliptic curves}\\
\text{- Discrete logarithm}\\
\text{- Algorithmic complexity}\\
\text{- Numerical computation}
\end{minipage}

\section*{Bibliographie commentée}

    De par l'essor des flux d'informations sensibles qu'a apporté
l'ère numérique, la cryptographie s'est révélée être un domaine 
essentiel à la sécurité des communications. De plus ces encrytions et 
décryptions se doivent d'être rapides et peu coûteuses à mettre en place.\\

    Un de ces systèmes, introduit en 1984 par Taher Elgamal~\cite{elgamal} , le cryptosystème d'ElGamal 
est un protocole de cryptographie asymétrique largement utilisé et 
construit sur le problème du logarithme discret \cite{dlp} . Cet algorithme permet 
à deux parties d'échanger un message de manière sécurisée en utilisant 
une paire de clés : une clé publique et une clé privée qui servent
respectivement à encrypter le message et à le décrypter.\\

    Cependant, les meilleurs algorithmes connus pour résoudre
le problème du logarithme discret sur les corps de nombres tels que le 
crible généralisé ont une complexité sous-exponentielle \cite{inriawr} , tandis que les
meilleurs algorithmes connus pour les groupes construits sur les courbes elliptiques \cite{CE} , comme l'algorithme de
rho de Pollard \cite{rdp} , ont une complexité exponentielle.\\

    C'est pourquoi en 2005 la National Security Agency (NSA) des États-Unis a 
recommandé l'utilisation de courbes elliptiques pour les systèmes
de cryptographie à clé publique, soulignant leur efficacité 
et leur sécurité accrues par rapport aux méthodes traditionnelles
\cite{nsa2005}. \\

    Le principe de l'encryptage par la méthode d'ElGamal repose donc sur le logarithme discret : il 
est nettement plus facile de calculer le reste de la division euclidienne 
que la réciproque de cette opération quand on opère sur un groupe cyclique.
Dans le cadre des courbes elliptiques ce groupe utilisé est construit à partir 
de l'ensemble des points d'une courbe elliptique dans le plan projectif 
sur lesquels on applique un logarithme et auxquels on ajoute un 
point à l'infini qui va servir de neutre pour la loi de groupe que l'on construit.\\

    Une première approche pour casser cette encryption serait d'essayer toutes les 
valeurs possibles de la clé privée jusqu'à trouver la bonne, d'opérer en force brute.
Cependant, la taille des clés utilisées dans les systèmes modernes rend cette approche
impraticable car le nombre de possibilités serait astronomique. Pour des clés de 256 bits,
il y aurait $2^{256}$ possibilités, ce qui est bien au-delà de la capacité de calcul 
de n'importe quel ordinateur. \\

    Pour réduire les temps de calcul au maximum, nous allons nous intéresser à l'algorithme
dit du "rho de Pollard" qui est un algorithme probabiliste qui repose sur le paradoxe 
des anniversaires \cite{anniversaires} et de la reconnaissance de cycles dans l'apparition des valeurs. 
Il est nettement plus efficace pour résoudre le problème du logarithme discret, 
il permet de trouver la clé privée en un temps approximativement proportionnel à la racine carrée
de l'ordre du groupe.\\
%    Dans le cadre des corps de nombres, le crible généralisé ....(si on a le temps d'investiguer cet algo mais ça semble compliqué à implémenter)

    Pour conclure, les courbes elliptiques offrent une sécurité supplémentaire
comparée aux groupes cycliques usuels utilisés dans le cryptosystème d'El Gamal.
Leurs structures mathématiques complexes et leurs mises en place relativement aisées 
offrent une solution de cryptographie robuste et efficace, adaptée à l'échange de clés
de sécurité dans des contextes de confidentialité très variés.

\section*{Problématique retenue}

En quoi les courbes elliptiques sont des objets mathématiques indispensables à la cryptographie moderne ?

\section*{Objectifs du TIPE}
\begin{itemize}
    \item construction de groupes cycliques sur les courbes elliptiques
    \item mise en place du cryptosystème d'El Gamal sur les courbes elliptiques
    \item comparaison de différents algorithmes de résolution du problème du logarithme discret
    \item étude comparative de la sécurité du cryptosystème d'El Gamal sur les courbes elliptiques et sur d'autres groupes cycliques usuels
\end{itemize}

\begin{thebibliography}{9}

\bibitem{elgamal}
ElGamal, T. (1985). A Public Key Cryptosystem and a Signature Scheme Based on Discrete Logarithms. In: Blakley, G.R., Chaum, D. (eds) Advances in Cryptology. CRYPTO 1984. Lecture Notes in Computer Science, vol 196. Springer, Berlin, Heidelberg.\href{https://doi.org/10.1007/3-540-39568-7_2}{DOI} 

\bibitem{dlp}
Aude LE GLUHER encadrée par Guénaël RENAULT 22 août 2015\href{https://perso.eleves.ens-rennes.fr/people/aude.legluher/fr/rapportl3.pdfURL}{Problème du logarithme discret appliqué à la
cryptanalyse sur courbes elliptiques : algorithme MOV}

\bibitem{inriawr}
Fabrice Boudot, Pierrick Gaudry, Aurore Guillevic, Nadia Heninger, Emmanuel Thomé, et al.. Nouveaux records de factorisation et de calcul de logarithme discret. Techniques de l’Ingénieur, 2021,
pp.17. ff10.51257/a-v2-in131ff. ffhal-03045666f \href{https://inria.hal.science/hal-03045666/document}{URL}

\bibitem{CE}
Elliptic Curves in cryptography, London Mathematical Society Lecture Notes Series 265, Cambridge university press \href{https://books.google.fr/books?id=0_vegzgyqGMC&printsec=frontcover&redir_esc=y#v=onepage&q&f=false}{URL}

\bibitem{rdp}
Handbook of Applied Cryptography, by A. Menezes, P. van
Oorschot, and S. Vanstone, CRC Press, 1996. \href{https://cacr.uwaterloo.ca/hac/about/chap3.pdf}{URL}

\bibitem{nsa2005}
National Security Agency (NSA). (2005). \textit{The Case for Elliptic Curve Cryptography}. Suite B Cryptography. \href{https://web.archive.org/web/20150815072948/https://www.nsa.gov/ia/programs/suiteb_cryptography/index.shtml}{URL(web archive)}

\bibitem{anniversaires}
wikipedia \href{https://fr.wikipedia.org/wiki/Paradoxe_des_anniversaires}{Paradoxe des anniversaires }

\end{thebibliography}

\end{document}
